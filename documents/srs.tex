\documentclass{article}
\usepackage[utf8]{inputenc}
\usepackage{caption}
\usepackage[margin=1in]{geometry}
\title{System Requirements Specification}
\author{Team EVAL}
\date{October 26, 2018}

\begin{document}

\maketitle

\newpage

\begin{center}
[Put team logo here]\\ \bigskip
{\LARGE College Course Evaluation System }\\ \medskip
{\large System Requirements Specification }\\
\end{center}

\tableofcontents

\newpage

\section{Introduction}
\subsection{Purpose of This Document}

This system requirements specification details what our course evaluation system does and what tests we will make to ensure the system is complete. It includes why we are creating the system, the scope of the product, diagrams that illustrate the system, what we will deliver to the customer, and currently pending issues. This document is intended for the product client, Harlan Onsrud, and potential users of the system.

\subsection{References}
\subsection{Purpose of the Product}

The University of Maine gives out course evaluation surveys to students at the end of each course. The survey is filled on a bubble sheet and is then scanned. Harlan Onsrud finds it inconvenient for the school administrators to manually scan and compile the survey results. He desires an online, automated evaluation system to improve productivity.\par

With this new product, course teachers and administrators can customize their lists of questions for course evaluations and send the lists out to students. Students will receive an e-mail telling them to complete the survey, and they will be periodically notified if the survey is incomplete. After survey submissions, the teachers and administrators can view the average of the results for each question over a certain course, instructor, department, and university.

\subsection{Product Scope}

\section{Functional Requirements}

The functional requirements specify what actions the program will perform. Each requirement is represented as a use case.

\begin{center}
\vspace{4in}
\captionof{table}{}
\begin{tabular}{|p{3.5cm}|p{7.5cm}|} 
\hline
\textbf{Number} & 1  \\
\hline
\textbf{Name} & Set up course  \\ 
\hline
\textbf{Summary} & An administrator or teacher adds a course to be surveyed along with its student roll \\ 
\hline
\textbf{Priority} & 5\\ 
\hline
\textbf{Preconditions }& The course's evaluation survey is unwritten \\ 
\hline
\textbf{Postconditions} & The course's evaluation survey is written, A student roll is linked to the course \\ 
\hline
\textbf{Primary Actors }& Administrator, Teacher \\ 
\hline
\textbf{Secondary Actors} & Database \\ 
\hline
\textbf{Trigger }& A new course needs to be evaluated \\ 
\hline
\textbf{Main Scenario }& 
\begin{tabular}{l|p{5.8cm}} 
\textbf{Step }& \textbf{Action}\\
\hline
1 & Select option to add a course \\
\hline
2 & Enter the course name and \newline corresponding student roster\\
\hline
3 & Database updated with course name and roster\\
\end{tabular}\\ 
\hline
\textbf{Extensions }&
\begin{tabular}{l|p{5.8cm}} 
\textbf{Step }& \textbf{Branching Action}\\
\hline
2a & No students are entered for a course : Notify user of the problem  \\
\end{tabular}\\
\hline
\textbf{Open Issues} & What information about the course shall be stored \\ 
\hline
\end{tabular}

\bigskip
\captionof{table}{}
\begin{tabular}{|p{3.5cm}|p{7.5cm}|} 
\hline
\textbf{Number} & 2  \\
\hline
\textbf{Name} & Edit survey questions  \\ 
\hline
\textbf{Summary} & An administrator or teacher edits the survey questions for a course \\ 
\hline
\textbf{Priority} & 5\\ 
\hline
\textbf{Preconditions }& A course for the survey questions exists in the database \\ 
\hline
\textbf{Postconditions} & A course includes its survey questions \\ 
\hline
\textbf{Primary Actors }& Administrator, Teacher \\ 
\hline
\textbf{Secondary Actors} & Database \\ 
\hline
\textbf{Trigger }& A college course has zero or wrong evalution survey questions \\ 
\hline
\textbf{Main Scenario }& 
\begin{tabular}{l|p{5.8cm}} 
\textbf{Step }& \textbf{Action}\\
\hline
1 & Choose the course whose questions are to be edited \\
\hline
2 & Select questions preset if applicable\\
\hline
3 & Enter the questions\\
\hline
4 & Questions stored in database and as text file\\
\end{tabular}\\ 
\hline
\textbf{Extensions }&
\begin{tabular}{l|p{5.8cm}} 
\textbf{Step }& \textbf{Branching Action}\\
\hline
3a & Invalid questions submitted : Notify user of invalid questions  \\
\end{tabular}\\
\hline
\textbf{Open Issues} & In what format shall the questions be stored in the text file \\ 
\hline
\end{tabular}

\bigskip
\vspace{2.6in}
\captionof{table}{}
\begin{tabular}{|p{3.5cm}|p{7.5cm}|} 
\hline
\textbf{Number} & 3  \\
\hline
\textbf{Name} & Create questions list preset  \\ 
\hline
\textbf{Summary} & An administrator or teacher makes a preset list of questions for a survey\\
\hline
\textbf{Priority} & 4\\ 
\hline
\textbf{Preconditions }& A course for the survey questions exists in the database \\ 
\hline
\textbf{Postconditions} & A preset list is available while creating a survey \\ 
\hline
\textbf{Primary Actors }& Administrator, Teacher \\ 
\hline
\textbf{Secondary Actors} & Database \\ 
\hline
\textbf{Trigger }& A user wants the same questions to be in multiple surveys \\ 
\hline
\textbf{Main Scenario }& 
\begin{tabular}{l|p{5.8cm}} 
\textbf{Step }& \textbf{Action}\\
\hline
1 & Select option to add a preset questions list \\
\hline
2 & Add the questions to the preset\\
\hline
3 & Database updated with preset\\
\end{tabular}\\ 
\hline
\textbf{Extensions }&
\begin{tabular}{l|p{5.8cm}} 
\textbf{Step }& \textbf{Branching Action}\\
\hline
2a & Questions in preset are invalid : Notify user of invalid questions  \\
\end{tabular}\\
\hline
\textbf{Open Issues} & Which presets are visible to each teacher \\ 
\hline
\end{tabular}

\bigskip
\captionof{table}{}
\begin{tabular}{|p{3.5cm}|p{7.5cm}|} 
\hline
\textbf{Number} & 4  \\
\hline
\textbf{Name} & Notify students  \\ 
\hline
\textbf{Summary} & A teacher e-mails students about the survey, giving automatic reminders to complete it \\ 
\hline
\textbf{Priority} & 5\\ 
\hline
\textbf{Preconditions }& A course evaluation survey is completed \\ 
\hline
\textbf{Postconditions} & All students are finished with their survey \\ 
\hline
\textbf{Primary Actors }& Teacher \\ 
\hline
\textbf{Secondary Actors} & Student, LimeSurvey \\ 
\hline
\textbf{Trigger }& The teacher releases the survey to students \\ 
\hline
\textbf{Main Scenario }& 
\begin{tabular}{l|p{5.8cm}} 
\textbf{Step }& \textbf{Action}\\
\hline
1 & Send initial reminder to students \newline about survey \\
\hline
2 & Students complete survey\\
\hline
3 & Copy LimeSurvey results to database\\
\end{tabular}\\ 
\hline
\textbf{Extensions }&
\begin{tabular}{l|p{5.8cm}} 
\textbf{Step }& \textbf{Branching Action}\\
\hline
2a & A student does not complete survey  : Send another reminder to student  \\
\end{tabular}\\
\hline
\textbf{Open Issues} & How often to remind students, handling signed comments \\ 
\hline
\end{tabular}

\bigskip
\vspace{2.6in}
\captionof{table}{}
\begin{tabular}{|p{3.5cm}|p{7.5cm}|} 
\hline
\textbf{Number} & 5 \\
\hline
\textbf{Name} & View survey results  \\ 
\hline
\textbf{Summary} & An administrator or teacher views averages of the survey results \\ 
\hline
\textbf{Priority} & 4\\ 
\hline
\textbf{Preconditions }& Survey responses are entered in database \\ 
\hline
\textbf{Postconditions} & All appropriate averages are computed and displayed \\ 
\hline
\textbf{Primary Actors }& Administrator, Teacher \\ 
\hline
\textbf{Secondary Actors} & Database \\ 
\hline
\textbf{Trigger }& A user seeks information about the responses to a group of surveys \\ 
\hline
\textbf{Main Scenario }& 
\begin{tabular}{l|p{5.8cm}} 
\textbf{Step }& \textbf{Action}\\
\hline
1 & Select option to view results \\
\hline
2 & Compute average score for each question \\
\hline
3 & Group averages by instructor, department, and university \\
\hline
4 & Display averages to user\\
\end{tabular}\\ 
\hline
\textbf{Open Issues} & Which survey questions are averaged for display \\ 
\hline
\end{tabular}
\end{center}

\subsection{Tests}

These are the tests that will verify the functional requirements:

\begin{enumerate}
  \item (Use Case 1) ...
  \item (Use Case 2) ...
  \item (Use Case 3) ...
  \item (Use Case 4) ...
  \item (Use Case 5) ...
\end{enumerate}

\section{Non-Functional Requirements}

The non-functional requirements state the qualities of the program that are unrelated to its function.

\begin{center}
\captionof{table}{}
\begin{tabular}{|p{3.5cm}|p{7.5cm}|} 
\hline
\textbf{Number} & 1  \\
\hline
\textbf{Priority} & 3\\ 
\hline
\textbf{Description} & The software should be supported by Windows, Mac, Linux, iOS, and Android. \\ 
\hline
\textbf{Tests }& 1 \\ 
\hline
\end{tabular}

\bigskip
\captionof{table}{}
\begin{tabular}{|p{3.5cm}|p{7.5cm}|} 
\hline
\textbf{Number} & 2  \\
\hline
\textbf{Priority} & 4\\ 
\hline
\textbf{Description} & The software should be accessible through Safari, Chrome, Firefox, and Edge. \\ 
\hline
\textbf{Tests }& 2 \\ 
\hline
\end{tabular}


\bigskip
\captionof{table}{}
\begin{tabular}{|p{3.5cm}|p{7.5cm}|} 
\hline
\textbf{Number} & 3  \\
\hline
\textbf{Priority} & 5\\ 
\hline
\textbf{Description} & All questions entered by the teacher or administrator shall appear on the output survey. \\ 
\hline
\textbf{Tests }& 3 \\ 
\hline
\end{tabular}


\bigskip
\captionof{table}{}
\begin{tabular}{|p{3.5cm}|p{7.5cm}|} 
\hline
\textbf{Number} & 4  \\
\hline
\textbf{Priority} & 5\\ 
\hline
\textbf{Description} & All data stored in the program's database shall be valid. \\ 
\hline
\textbf{Tests }& 4 \\ 
\hline
\end{tabular}


\bigskip
\captionof{table}{}
\begin{tabular}{|p{3.5cm}|p{7.5cm}|} 
\hline
\textbf{Number} & 5  \\
\hline
\textbf{Priority} & 5\\ 
\hline
\textbf{Description} & All collected survey data shall not be alterable.\\ 
\hline
\textbf{Tests }& 4 \\ 
\hline
\end{tabular}


\bigskip
\captionof{table}{}
\begin{tabular}{|p{3.5cm}|p{7.5cm}|} 
\hline
\textbf{Number} & 6  \\
\hline
\textbf{Priority} & 4\\ 
\hline
\textbf{Description} & Teachers shall not be able to access data of courses other than their own. \\ 
\hline
\textbf{Tests }& 5 \\ 
\hline
\end{tabular}


\bigskip
\captionof{table}{}
\begin{tabular}{|p{3.5cm}|p{7.5cm}|} 
\hline
\textbf{Number} & 7  \\
\hline
\textbf{Priority} & 3\\ 
\hline
\textbf{Description} & The mean time between failures should be at least 60 minutes. \\ 
\hline
\textbf{Tests }& 6 \\ 
\hline
\end{tabular}


\bigskip
\captionof{table}{}
\begin{tabular}{|p{3.5cm}|p{7.5cm}|} 
\hline
\textbf{Number} & 8  \\
\hline
\textbf{Priority} & 5\\ 
\hline
\textbf{Description} & Students shall have no access to any data stored by the program. \\ 
\hline
\textbf{Tests }& 7 \\ 
\hline
\end{tabular}


\bigskip
\captionof{table}{}
\begin{tabular}{|p{3.5cm}|p{7.5cm}|} 
\hline
\textbf{Number} & 9  \\
\hline
\textbf{Priority} & 5 \\ 
\hline
\textbf{Description} & All survey responses shall be anonymous. \\ 
\hline
\textbf{Tests }& 5 \\ 
\hline
\end{tabular}


\bigskip
\captionof{table}{}
\begin{tabular}{|p{3.5cm}|p{7.5cm}|} 
\hline
\textbf{Number} & 10  \\
\hline
\textbf{Priority} & 2\\ 
\hline
\textbf{Description} & The software should scale to at least three universites, 1000 courses per semester, 1000 teachers per university, and 500 students per course. \\ 
\hline
\textbf{Tests }& 8 \\ 
\hline
\end{tabular}


\bigskip
\captionof{table}{}
\begin{tabular}{|p{3.5cm}|p{7.5cm}|} 
\hline
\textbf{Number} & 11  \\
\hline
\textbf{Priority} & 1 \\ 
\hline
\textbf{Description} & The software should not exceed 500 MB in size. \\ 
\hline
\textbf{Tests }& 9 \\ 
\hline
\end{tabular}


\bigskip
\captionof{table}{}
\begin{tabular}{|p{3.5cm}|p{7.5cm}|} 
\hline
\textbf{Number} & 12  \\
\hline
\textbf{Priority} & 4\\ 
\hline
\textbf{Description} & The software's source code shall be open-source. \\ 
\hline
\textbf{Tests }& 10 \\ 
\hline
\end{tabular}


\bigskip
\captionof{table}{}
\begin{tabular}{|p{3.5cm}|p{7.5cm}|} 
\hline
\textbf{Number} & 13  \\
\hline
\textbf{Priority} & 4 \\ 
\hline
\textbf{Description} & The licensing requirements of any non-original code shall be met.\\ 
\hline
\textbf{Tests }& 11 \\ 
\hline
\end{tabular}

\bigskip
\vspace{.5in}
\captionof{table}{}
\begin{tabular}{|p{3.5cm}|p{7.5cm}|} 
\hline
\textbf{Number} & 14  \\
\hline
\textbf{Priority} & 4 \\ 
\hline
\textbf{Description} & The software shall meet UMaine AFUM requirements.\\ 
\hline
\textbf{Tests }& 11 \\ 
\hline
\end{tabular}

\end{center}

\subsection{Tests}

These are the tests that will verify the non-functional requirements:

\begin{enumerate}
  \item ...
  \item ...
  \item ...
  \item ...
  \item ...
  \item ...
  \item ...
  \item ...
  \item ...
  \item ...
  \item ...
\end{enumerate}

\section{User Interface}

See ``User Interface Design Document for the College Course Evaluation System.''

\section{Deliverables}

The following lists the date and format that each submission will be delivered:

\begin{center}
\captionof{table}{}
\begin{tabular}{|p{6cm}|p{3cm}|p{3cm}|} 
\hline
\textbf{Submission} & \textbf{Date of Delivery} & \textbf{Format} \\
\hline
System Requirements Specification & 10/26/2018 & Hard Copy\\ 
\hline
System Design Document & ?/??/2018 & Hard Copy\\ 
\hline
User Interface Design Document & ?/??/2018 & Hard Copy\\ 
\hline
User Manual & ?/??/2019 & Hard Copy\\ 
\hline
Administration Manual & ?/??/2019 & Hard Copy\\ 
\hline
System Requirements Specification & ?/??/2019 & Electronic\\ 
\hline
System Design Document & ?/??/2019 & Electronic\\ 
\hline
User Interface Design Document & ?/??/2019 & Electronic\\ 
\hline
User Manual & ?/??/2019 & Electronic\\ 
\hline
Administration Manual & ?/??/2019 & Electronic\\ 
\hline
Source Code & ?/??/2019 & Electronic\\ 
\hline
Web Link to Program & ?/??/2019 & Electronic\\ 
\hline
\end{tabular}
\end{center}

\section{Open Issues}

The most significant issue our team has relates to the non-anonymous student responses. In a course evaluation survey, a student has the option to add a signed comment to be stored in an instructor's file. However, the LimeSurvey software does not store the identities of survey respondents. We need to find a way to collect the signed comment along with the survey, while ensuring that the signature is authentic.

There are several more minor issues that need to be resolved. We have not decided what information about each course should be stored in the database. We do not know what format to use for the questions file and how to import it into LimeSurvey. We should also know which question presets and which surveys' results are visible to each user. Finally, we should decide how often to notify students about their survey.

\appendix

\newpage
\section{Agreement Between Customer and Contractor}

\vspace{.7in}
\noindent
\begin{tabular}{ p{5cm} p{5cm} p{5cm} } 
\textbf{\textit{Name}} & \textbf{\textit{Signature}} & \textbf{\textit{Date}} \\[.5cm]
\textbf{Jovon Craig} & $\rule{5cm}{.1mm}$ & $\rule{5cm}{.1mm}$\\[.5cm]
\textbf{Sam Elliot} & $\rule{5cm}{.1mm}$ & $\rule{5cm}{.1mm}$\\[.5cm]
\textbf{Robert Judkins} & $\rule{5cm}{.1mm}$ & $\rule{5cm}{.1mm}$\\[.5cm]
\textbf{Stanley Small} & $\rule{5cm}{.1mm}$ & $\rule{5cm}{.1mm}$\\[.5cm]
\textbf{Harlan Onsrud} & $\rule{5cm}{.1mm}$ & $\rule{5cm}{.1mm}$\\[.5cm]
Customer Comments: & \multicolumn{2}{ l }{ $\rule{10.5cm}{.1mm}$ }\\[.5cm]
\multicolumn{3}{ l }{ $\rule{16cm}{.1mm}$ }\\[.5cm]
\end{tabular}

\newpage
\section{Team Review Sign-off}

This page shows that all members of Team EVAL have agreed on the content of the system requirements specification. Each team member agrees on the goals of the project, each use case and requirement, and how the deliverables will be sent. There is nothing in the document that is a source of contention.

\vspace{.7in}
\noindent
\begin{tabular}{ p{5cm} p{5cm} p{5cm} } 
\textbf{\textit{Name}} & \textbf{\textit{Signature}} & \textbf{\textit{Date}} \\[.5cm]
\textbf{Jovon Craig} & $\rule{5cm}{.1mm}$ & $\rule{5cm}{.1mm}$\\[.5cm]
Comments: & \multicolumn{2}{ l }{ $\rule{10.5cm}{.1mm}$ }\\[.5cm]
\multicolumn{3}{ l }{ $\rule{16cm}{.1mm}$ }\\[.5cm]
\textbf{Sam Elliot} & $\rule{5cm}{.1mm}$ & $\rule{5cm}{.1mm}$\\[.5cm]
Comments: & \multicolumn{2}{ l }{ $\rule{10.5cm}{.1mm}$ }\\[.5cm]
\multicolumn{3}{ l }{ $\rule{16cm}{.1mm}$ }\\[.5cm]
\textbf{Robert Judkins} & $\rule{5cm}{.1mm}$ & $\rule{5cm}{.1mm}$\\[.5cm]
Comments: & \multicolumn{2}{ l }{ $\rule{10.5cm}{.1mm}$ }\\[.5cm]
\multicolumn{3}{ l }{ $\rule{16cm}{.1mm}$ }\\[.5cm]
\textbf{Stanley Small} & $\rule{5cm}{.1mm}$ & $\rule{5cm}{.1mm}$\\[.5cm]
Comments: & \multicolumn{2}{ l }{ $\rule{10.5cm}{.1mm}$ }\\[.5cm]
\multicolumn{3}{ l }{ $\rule{16cm}{.1mm}$ }\\[.5cm]
\end{tabular}


\newpage
\section{Document Contributions}
\end{document}

